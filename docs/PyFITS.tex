\documentstyle[11pt,adassconf]{article}


\begin{document}
\paperID{P7.3}
\contact{Paul E. Barrett}
\email{barrett@compass.gsfc.nasa.gov}
\paindex{Barrett, P.E.}
\aindex{Bridgman, W.T.}
\keywords{python, FITS utilities, data formats}

\title{PyFITS, a FITS Module for Python}
\author{P.E. Barrett\altaffilmark{1}}
\affil{NASA/Goddard Space Flight Center, Greenbelt, MD 20771} 
\author{W.T. Bridgman\altaffilmark{1}}
\affil{NASA/Goddard Space Flight Center, Greenbelt, MD 20771} 
\altaffiltext{1}{Universities Space Research Association, Seabrook, MD}

\begin{abstract}

PyFITS is a module for reading, writing, and manipulating FITS files
using the interactive, object-oriented language, Python.  The module
is composed of two files: a generic low-level C library for
manipulating multidimensional arrays of C-type structures and a high-level 
Python module.  FITS
files can be manipulated at several different levels, beginning with
the header-data unit at the highest level to rows and columns of
binary tables at the lowest level.  In addition, header-data units and
columns of binary tables are accessible by index or name. PyFITS also
interfaces to NumPy, the Python numerical array module.
\end{abstract}

\section{What is Python?}

{\bf python}, ({\it Gr. Myth}.  An enormous serpent that lurked in the
cave of Mount Parnassus and was slain by Apollo).  1. any of a genus
of large, non-poisonous snakes of Asia, Africa and Australia that
crush their prey to death.  2. popularly, any large snake that crushes
its prey.  3. totally awesome, bitchin' language that will someday
crush the \$s out of certain other so-called VHLL's ;-) %$

Python is an interpreted, interactive, object-oriented programming
language.  It can be used either interactively, by typing {\tt python} at the 
shell prompt, or as a script, by typing {\tt python somescript.py} (It is often 
compared to Tcl, Perl, Scheme or Java).  It combines 
remarkable power with very clear syntax and has
modules, classes, exceptions, very high level dynamic data types, and
dynamic typing.  There are interfaces to many system calls and
libraries, as well as to various windowing systems (X11, Motif, Tk,
Mac, MFC).  New built-in modules are easily written in C or C++.
Python is also usable as an extension language for applications that
need a programmable interface.


The Python implementation is portable: it runs on many brands of UNIX,
on Windows, DOS, OS/2, Mac, Amiga...  If your favorite system isn't
listed here, it may still be supported, if there's a C compiler for
it.  Ask around on the {\tt comp.lang.python} newsgroup -- or just try compiling Python
yourself.  Python is copyrighted but freely usable and distributable, even for
commercial use.  Access to the source code can be found through the 
\htmladdnormallink{Python Home Page}{http://www.python.org/}.  For more information on 
extending Python, see Pirzkal (1999) in these proceedings.


\section{What is PyFITS?}

PyFITS is python module for reading, writing, and manipulating FITS
files.  A FITS file is treated as a 'list' of {\em header-data units}
or HDUs.  Access to the HDU is by an integer index or using the
extension name as a dictionary key.  The primary HDU can be accessed
using the keyword 'PRIMARY'.  This keyword or dictionary feature of
PyFITS enables access to a HDU without having to know the absolute
index of the extension in the FITS file.  Hence, software which uses
this feature of PyFITS will be more robust and portable than
software that does not.

Each HDU contains two parts, a {\em header} and {\em data} unit,
though the data unit may contain no data.  The header part of the HDU
is a list of records containing the keyword, the value, and a
comment.  Access to the header unit is via list behavior, so records
can be accessed, assigned, appended, deleted, inserted, etc. by using
either its integer index or its keyword.  The benefits of keyword
access to the header are as noted above.

The data part of the HDU is treated as an array or a list of records
depending on the extension type.  For example, the primary array
obviously has array behavior, while the binary table extension is
treated as a 1-dimensional array of C-structures or records.  It is
therefore possible to access an entire array or extension or just a
small slice.  For example when accessing a binary table extension, a
single record or list of records can be accessed using array notation.
It is also possible to access just a single column of a binary table
by using an integer index or the column name.  All of these features
allow quick and efficient access to all aspects of the data in a FITS
file.

The listing below is a sample session demonstrating the use of Python 
with the PyFITS module.


%\footnotesize
\begin{verbatim}

% python
>>> from fitsio import *
>>> fits = Ffile('a_fits_file.fits', 'r')
>>> len(fits)                   # number of HDUs in file
4
>>> fits[0].hdr                 # print primary header
SIMPLE  =                    T  /  FITS STANDARD                           
BITPIX  =                   16  /  Binary Data                             
NAXIS   =                    2  /  Number of image axes                    
NAXIS1  =                  512  /  Dimension of axis 1                     
NAXIS2  =                  512  /  Dimension of axis 2                     
BSCALE  =           1.000000E0  /  Real = tape*BSCALE + BZERO              
BZERO   =           0.000000E0  /  Real = tape*BSCALE + BZERO              
DATE    = '10/05/93'  /  FITS creation date                                
IRAFNAME= './rp30019301_im1.imh'  /  Output IRAF image name                
TELESCOP= 'ROSAT   '  /  telescope (mission) name                          
INSTRUME= 'PSPC    '  /  instrument (detector) name                        
RADECSYS= 'FK5     '  /  WCS for this file (e.g. Fk4)                      
EQUINOX =           2.000000E3  /  equinox (epoch) for WCS                 
CTYPE1  = 'RA---TAN'  /  axis type for dim. 1 (e.g. RA---TAN)              
CTYPE2  = 'DEC--TAN'  /  axis type for dim. 2 (e.g. DEC--TAN)              
...
END 
>>> fits[0].hdr['BITPIX']       # print 'BITPIX' value
16
>>> a = fits[0].data            # assign primary array to 'a'
>>> 
>>> x = fits[1].data[:][0]      # assign column 1 of binary table
>>>                             #   to array x
>>> event = fits['EVENT']       # assign EVENT extension to 
>>>                             #   variable event
>>> y = event[:]['Y']           # assign 'Y' column of binary 
>>>                             #   table 'EVENTS' to array y
>>> 
\end{verbatim}

This sample demonstrates opening a FITS file and displaying the header 
keywords, along with the basic syntax of access the binary data tables 
in the file.

\section{Implementing PyFITS}

PyFITS is mostly coded in Python using Python classes, since this
allows for the fastest development and most portability.  PyFITS
contains 12 classes: FBool, Fcard, Fhead, Fdata, Ffield, FHDU,
FPrimary, FRandomGroupHDU, FASCIITableHDU, FBinTableHDU, FITS, and
Ffile.  FBool, Fcard, Fhead, Fdata, and Ffield are low level classes
and provide access to header card records, data arrays, etc..  The
FHDU class is a base header-data unit class.  The other HDU classes
inherit the basic behavior of the FHDU class.  The highest level class
is the FITS class itself.  An associated class that allows file access
is the Ffile class.  Note that a FITS object does not necessarily have 
to be associated with a FITS file.  The file is only used for
permanent storage.

The one part of FITS that has not been coded in Python is the array
record module.  This module is imported by PyFITS and is used to read
and write the binary data.  Python did not have an efficient mechanism
at the time to read and write binary data or records and a Python
implementation, though usable, was not very fast or efficient.  Hence,
a C-extension module was written to read and write binary records from
native machine format to big- or little-endian byte formats.  This is
a general purpose module capable of reading multi-dimensional arrays of
C-structures or records.  The module can be used independently of
PyFITS to read and write binary data and has links to import and
export data to the NumPy module for matrix and array math.

Since this module is fully integrated into Python, it allows efficient
access to a binary data file using array notation.  It is possible to
access the data as a single record, a list of records, a single item, a
list of items, or any combination of these.  The module also allows
items in a record to be named and then accessed by its name, instead
of by its column or field index.

\section{Conclusion}

PyFITS is a Python extension module which enables astronomers to 
easily and efficiently manipulate FITS files either interactively 
using the Python command-line prompt, or as part of a large executable 
program.  Most of the program is coded in Python, making it easy to 
enhance and maintain, while the data access layer has been coded in 
C, making it fast and efficient.  A possible next step in the 
development of this project would be to add a GUI for file browsing.

The development of PyFITS provides a good example of modern 
programming principles and design.  The implementation uses 1) an 
object-oriented programming language which enables a modular design, 
reuse of code via inheritance, and operator overloading; 2) a 
very-high-level language (VHLL), like Python, for rapid program 
development (development times are typically a factor of ten faster 
than using compiled languages), while using a low-level language, 
like C, for speed and efficiency (only for those parts of the code 
that really need it), and 3) an interpreted (scripting) language for 
fast development and ease of use.

\begin{references}
\reference Lutz, M.\ 1996, ``Programming Python'', O'Reilly and Associates, Inc.
\reference Dubois, P.\ 1996, Computers in Physics, May/June issue
\reference Python Home Page http://www.python.org/
\reference Pirzkal, N.\ \& Hook, R.\ 1999, \adassviii, \paperref{P7.12}
\end{references}

\end{document}

% =============================================================================
